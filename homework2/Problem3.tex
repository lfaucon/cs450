\paragraph{1.}
Let use the strategy where at any step $n$ ($n$ starts at $0$) the position of the dog is given by $x_n = (-2)^n$ (except if the dog has found the bone, in this case the dog stops at the bone).
\newline
\newline
The total distance $t_n$ walked  by the dog since the beginning of the search until the step $n$ is given by the relation
\newline
$t_n = \begin{cases}
1 & \mbox{if } n \mbox{ is 0}
\\ t_{n-1}+2^{n-1}+2^{n}=t_{n-1}+3*2^{n-1} & \mbox{if } n>0 \mbox{, bone not found at step n}
\\ t_{n-1}+2^{n-1}+|x| & \mbox{if the bone was found at step } n
\end{cases}$
\newline
\newline
Thus, we have
\newline
$t_n=\begin{cases}
3*(2^n-1)+1 & \mbox{for every } n \mbox{ before the dog finds the bone}
\\ 3*(2^{n-1}-1) + 2^{n-1} + |x| & \mbox{at the step where the dog finds the bone}
\end{cases}$
\newline
(Even if the bone is found at the step $0$).
\subparagraph{}
If $x > 0$ or $x < 0$, the dog finds the bone at the step $N$ such that $2^{N-2}<|x|\leq 2^N$ (the dog didn't find the bone at the second to last step).
\newline
Then in this case the total walked distance is:
\newline
$T=t_N=3*(2^{N-1}-1)+1+2^{N-1}+|x|<3*(2*|x|-1) +1 + 2*|x|+|x|=9*|x|-2$
\subparagraph{}
Thus, the total walked distance is bounded by $9*|x|-2$ . Our algorithm is competitive with the best algorithm with a ratio of $9$.

\paragraph{2.}
This time we have $m>2$ roads crossing at the same point. We index the roads by $0..m-1$. We note by $q(n,m)$ the quotient of $n$ divided by $m$, and $r(n,m)$ the remainder of the division of $n$ by $m$.
\subparagraph{}
Our dog will search the bone by making circles, going from road $r$ to road $r+1[m]$ at each step. We are going to use quite the same strategy as before: we will double the distance each time the dog passes by the road $0$.
Let $X_n=(r_n,x_n)$ be the tuple containing the index of the road on which the dog is and his absciss on this road.
\newline
The dog will change of road between each step, and double the absciss if he is on the road $0$: so we have $r_n=r(n,m)$ and $x_n=2^{q(n,m)}$.
\subparagraph{}
Let $t_n$ be the distance that the dog has walked since the beginning of the search. We have the relation:
$t_n = \begin{cases}
1 & \mbox{if } n \mbox{ is 0}
\\ t_{n-1}+2^{q(n-1,m)}+2^{q(n,m)} & \mbox{if } n>0 \mbox{, bone not found at step n}
\\ t_{n-1}+2^{q(n-1,m)}+x & \mbox{if the bone was found at step } n
\end{cases}$
\newline
\subparagraph{}
At the step $n$, the dog has made $q(n,m)$ times a complete lap.
\begin{itemize}
\item To get at the origin (absciss $0$ on the road $0$) after the $q(n,m)$ laps, the dog has walked the distance:
\newline
$\sum\limits_{l=0}^{q(n,m)-1}{2*m*2^l} = 2m*(2^{q(n,m)}-1)$
\newline
(because there are $m$ roads, and the dog must walk back and forth on the roads).
\newline
\item Then, to be on the correct road, at the correct absciss, the dog must walk
\newline
$\begin{cases}
(2*r(n,m)+1)*2^{q(n,m)} & \mbox{(if he hasn't found the bone at the step } n
\\ 2*r(n,m)*2^{q(n,m)}+x  & \mbox{if he has found the bone at the step } n
\end{cases}$
\end{itemize}
So we have:
\newline
$t_n = \begin{cases}
2m*(2^{q(n,m)}-1) + (2*r(n,m)+1)*2^{q(n,m)} & \mbox{for every } n \mbox{ before the dog finds the bone}
\\ 2m*(2^{q(n,m)}-1) + 2*r(n,m)*2^{q(n,m)}+ x & \mbox{at the step where the dog finds the bone}
\end{cases}$
\subparagraph{}
Let suppose that the bone is on the road $r$, then the dog finds it at a step $N=m*k+r$ such that $2^{k-1}<x\leq 2^k$.
\newline
The distance that the dog has walked to find the bone is:
\newline
$T=t_N= 2m*(2^k-1) + 2r*2^{k}+ x<4mx - 2m +4rx +x=(4(m+r)+1)x-2m$
\newline
Thus, we always have $T=t_N<(8m-3)x-2m=O(mx)$.
\newline
\newline
If we use the fact that $x\leq 2^k$, we deduce that:
\newline
$T=t_N= 2m*(2^k-1) + 2r*2^{k}+ x \geq 2m(x-1)+2rx +x = (2(m+r)+1)x -2m \geq 2m(x-1)$.
So, $T=\Omega(mx)$
\subparagraph{}
As a conclusion, we have a competitive ratio of $\Theta(mx)$.