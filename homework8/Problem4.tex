We assume that we want to find the best path $Q$ such that the sum of delays is minimized given a list of points $[p_1,..,p_n]$ (sorted by absciss).

\subsection*{Some remarks about the meaningful paths}

A meaningful path $Q$ is given by the sequence in which we visit the points. In a meaningful path $Q$, the index $i$ of a point $p_i$ is written once and only once.
\newline
For such a path $Q$, $Q=q_1..q_n$ (where $\forall i, q_i \in \{1,..,n\}$) means: we first go to the point of index $q_1$, then $q_2$ and so on, until we finish by going to $q_n$.
\newline
A meaningful path $Q$ is such that when we go from the point of index $q_i$ to the point $q_{i+1}$, we have already been to the points $q \in \{min(q_i,q_{i+1}), ..., max(q_i, q_{i+1})-1\}$, i.e $\forall i \in \{1,..,n-1\}, \forall q \in \{min(q_i,q_{i+1}), ..., max(q_i, q_{i+1})-1\}, \exists j \in \{1,..,i-1\} such\  that q_j=q$.

If we look at how a meaningful path $Q$ can end, we can see that it cannot end by going to an interior point: the path must end either at $p_1$ or $p_n$ because if it is not the case, then we would have visited $p_1$ and $p_n$ before, thus violating our conventions for a meaningful path.

\subsection*{How to reduce the problem}

Therefore, we know that our best path $Q$ is the best solution between the solutions:
\begin{itemize}
\item $Q_{end\ right}=Q_{next\ is\ right\ \{1,..,n-1\}}n$ where $Q_{next\ is\ right\ \{1,..,n-1\}}$ is the meaningful path with minimal delay passing by $p_1,..,p_{n-1}$ and for which we know that we will end the path by going right at point of index $n$,
\item or the solution $Q_{end\ left}=Q_{next\ is\ left\ \{2,..,n\}}1$ where $Q_{next\ is\ left\ \{2,..,n\}}$ is the meaningful path with minimal delay passing by $p_2,..,p_n$ and for which we know that we will end the path by going left at point of index $1$
\end{itemize}

Thus, we want to be able to determine $Q_{next\ is\ right\ \{1,..,n-1\}}$ and $Q_{next\ is\ left\ \{2,..,n\}}$. This is easy by doing it recursively if we define $Q_{next\ is\ right\ \{i,..,j\}}$ $\forall 1 \leq i \leq j<n$ and $Q_{next\ is\ left\ \{i,..,j\}}$ $\forall 1<i \leq j \leq n$:
\begin{itemize}
\item if $i=j$ then $Q_{next\ is\ right\ \{i\}} = i$
\item $\forall i<j, Q_{next\ is\ right\ \{i,..,j\}}$ is the sub-path $Q_{next\ is\ right\ \{i,..,j-1\}}j$ or $Q_{next\ is\ left\ \{i+1,..,j\}}i$ (it is $Q_{next\ is\ right\ \{i,..,j-1\}}j$ if the path $Q_{next\ is\ right\ \{i,..,j-1\}}j(j+1)$ has the lowest total delay)    
\item if $i=j$ then $Q_{next\ is\ left\ \{i\}} = i$
\item $\forall i<j, Q_{next\ is\ left\ \{i,..,j\}}$ is the sub-path $Q_{next\ is\ right\ \{i,..,j-1\}}j$ or $Q_{next\ is\ left\ \{i+1,..,j\}}i$ (it is $Q_{next\ is\ right\ \{i,..,j-1\}}j$ if the path $Q_{next\ is\ right\ \{i,..,j-1\}}j(i-1)$ has the lowest total delay) 
\end{itemize}

With such definitions, $Q_{next\ is\ right\ \{i,..,j\}}$ is the best sub-path knowing that the next step is to go to the point of index $j+1$, while $Q_{next\ is\ left\ \{i,..,j\}}$ is the best sub-path knowing that the next step is to go to the point of index $i-1$.

\subsection*{Using dynamic programming}

We use $memoization$ to compute and store the different sub-paths $Q_{next\ is\ right\ \{i,..,j\}}, \forall 1 \leq i \leq j < n$ and $Q_{next\ is\ left\ \{i,..,j\}} \forall 1 < i \leq j \leq n$ effectively.
\newline
For that, we store the two types of sub-paths in two square matrices of size $n \times n$ each (but we use only the upper-left $\frac{n(n-1)}{2}$ elements in each matrix for example, because our sequences are such that $i<j$).
\newline
The algorithm that we have really implemented is recursive: we only compute a sub-path if it is not in the corresponding cache.
\newline
However, we can implement a fully sequential algorithm easily: we initialize first the sub-paths of length 1 with the point they contain and a zero delay; then, for each size of sequence $l$ (increasing), we compute the sub-paths thanks to the sub-paths of size $l-1$.

\subsection*{Running time and implementation}
The running time of the algorithm is therefore $\Theta(n^2)$ due to the size of our sub-paths caches and that to determine a bigger sub-path (either finishing on the right or on the left) we only have two possibilities of doing so.
Because we had some difficulties to find this algorithm, we have made an implementation of it. Feel free to ask it.