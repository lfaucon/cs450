\subparagraph{}
We consider a graph of $n$ nodes, for which a min-cut is of size $c$. We call $k$ the number of steps before forking the instances, and $m$ the number of different instances after the fork.
\newline
We look at $k$ and $m$ such that $k + m*(n-2-k) = 2*(n-2)$, meaning that the number of merging operations is equal to the number of operations with two instances of the traditional algorithm seen during the course.
\newline
At each step $i$, our algorithm (for each instance) pick an edge and merge the two vertices associated to the edge, creating a new graph with one less vertex.
\subparagraph{}
Because there may be numerous min-cuts, we will compare this algorithm with a chosen min-cut to make the following analysis.
We will find an upper bound of the probability that no-resulting cut of this algorithm is like the chosen min-cut. (This will provide us an upper bound of the probability that we don't have a correct min-cut at all).

\subparagraph{}
We call the probability $p$ that every resulting cut of the algorithm is not like the chosen min-cut. Then we have by definition of $p$:
\newline
\begin{tabular}{rl}
$p = $ & $Pr(every\: cut\: different\: of\: the\: chosen\: cut | cut\: correct\: at\: step\ k)Pr(cut\: correct\: at\: step\: k)$ \\
& $+ Pr(every\: cut\: different\: of\: the\: chosen\: cut | cut\: not\: good\: at\: step\: k)Pr(cut\: incorrect\: at\: step\: k)$\\
\end{tabular}

\subparagraph{}
We now will bound each term.

\begin{itemize}

\item We call $q$ the term $Pr(every\: cut\: different\: of\: the\: chosen\: cut | cut\: correct\: at\: step\ k)$.
\newline
We have:
\newline
$q=Pr(one\: cut\: different\: of\: the\: chosen\: cut| cut\: correct\: at\: step\ k)^m$
\newline
Because the evolution of each instance after the fork can be fairly assumed independent.
\newline
\newline
We call $q'$ the term $Pr(one\: cut\: different\: of\: the\: chosen\: cut| cut\: correct\: at\: step\ k)$.
\newline
We can thus bound $q'$ with the same reasoning as seen during the class, which provides us the bound:
\newline
$q' \leq 1 - \prod_{i=k}^{n-3}{1 - \frac{2}{n-i}} = 1 -  \prod_{i=k}^{n-3}{\frac{n-i-2}{n-i}} = 1 -  \frac{(n-2)!(n-k)!}{(n-k-2)!n!}$
\newline
Thus,
$q' \leq 1 - \frac{2}{(n-k)(n-k-1)}$
\newline
\newline
Which permits us to bound $q$ with:
\newline
$q \leq (1 - \frac{2}{(n-k)(n-k-1)})^m$

\item We call $r$ the term $Pr(cut\: correct\: at\: step\: k)$.
\newline
This term is a little bit difficult to bound efficiently.
\newline
Assuming that a graph of $m$ node is simple, we can be sure that there are no more than $\frac{m(m-1)}{2}$ edges in the graph. We will assume that the graph for which we are trying to find a min-cut is simple, so that at any step $i$ of the algorithm, the number of edges is at most $\frac{(n-i)(n-i-1)}{2}$.
\newline
Thus, at any step $i$ of the algorithm, the probability to pick an edge from the chosen min-cut is at least $\frac{c}{\frac{(n-i)(n-i-1)}{2}}$.
\newline
Hence, the probability to NOT pick an edge of the min-cut is at most $1-\frac{2c}{(n-i)(n-i-1)}$.
\newline
Which gives us:
\newline
$r \leq \prod_{i=0}^{k-1}{(1-\frac{2c}{(n-i)(n-i-1)})}$.
\newline

\item We call $t$ the term $Pr(every\: cut\: different\: of\: the\: chosen\: cut | cut\: not\: good\: at\: step\: k)$.
\newline
This term is equal to $1$ because of the condition implying the tested event.
\item We call $u$ the term $Pr(cut\: incorrect\: at\: step\: k)$.
\newline
We use the same argument as in the lectures notes to bound this term: because the min-cut is made of $c$ edges, we know that every vertex has a degree of at least $c$, hence the number of edges is at least $c(n-i)/2$ at a step $i$ of the algorithm.
\newline
Therefore, at any step $i$ of the algorithm, the probability to pick an edge from the chosen min-cut is at most $\frac{c}{\frac{c(n-i)}{2}}=\frac{2}{n-i}$.
\newline
Hence, at any step $i$ of the algorithm, the probability to NOT pick an edge from the chosen min-cut is at least $1-\frac{2}{(n-i)}$.
\newline
So that the probability that tha cut is correct at the stek $k$ is at least:
\newline
$\prod_{i=0}^{k-1}{(1-\frac{2}{(n-i)})} = \prod_{i=0}^{k-1}{\frac{n-i-2}{n-i}} = \frac{2}{(n-k)(n-k-1)}$
\newline
And the probability that a cut is incorrect at step $k$ is at most:
\newline
$u \leq 1 -  \frac{2}{(n-k)(n-k-1)}$.
\end{itemize}
\subparagraph{}
In the end, we have bounded $p$ by:
\newline
$p \leq (1 - \frac{2}{(n-k)(n-k-1)})^m \prod_{i=0}^{k-1}{(1-\frac{2c}{(n-i)(n-i-1)})} + 1 -  \frac{2}{(n-k)(n-k-1)}$
\subparagraph{}
This bound is really hard to make as little as possible because of the presence in the formula of the number of edges in the cut $c$, and the fact that when the term $q$ disminishes, the term $r$ grows and vice versa...
\newline
Hence, there is no way to find optimal numbers $k$ and $m$ without knowing $c$. If $c$ is known, such numbers $k$ and $m$ can be found trying to minimize numerically the bound we have explicited.
\newline
If $c$ is unknown, we can for example take $k \approx \frac{n-2}{2}$ and $m = 3$ by default. This choice is made so that the total number of merging operations is $2(n-2)$.