

\paragraph{1.} 
\paragraph{}
After 15 pushes followed by 7 pops the stack will be in the following state :

\begin{tabular}{rl}
$S_0$ & 8 \\
$S_1$ & 7 \\
$S_2$ & 6 5 \\
$S_3$ & 4 3 2 1 \\
\end{tabular}


\paragraph{2.} 
\paragraph{}
The worst case scenario for pop is for any given integer k when $S_0$ contains one element and for i from 1 to k $S_i$ is half full. In that state a pop will empty $S_0$ which will then take elements from $S_1$ emptying it in the process. $S_1$ will then take elements from $S_2$ emptying it in the process. This so empties every $S_i$ until the last one resulting in a number of operation equal to the number of elements in the stack. We can also realize that the resulting state is the worst case scenario for a push operation.
\paragraph{}
The worst case for push is for any given integer k when for i from 0 to k $S_i$ is completely full. In that state pushing an element in the stack will overload $S_0$ which will then push its elements on $S_1$ overloading it in the process. $S_1$ will then push its elements on $S_2$ overloading it in the process. This so overload every $S_i$ until the last one resulting in a number of operation equal to the number of elements in the stack. We can also realize that the resulting state is the worst case scenario for a pop operation.
\paragraph{}
The worst case scenario for pop and push are both in linear $O(n)$ time.


\paragraph{3.} 
\paragraph{}
Let $k$ be any integer and $n=2^k-1$. 

\paragraph{Initial state : } Empty stack
\paragraph{Operation sequence : }
\begin{enumerate}
\item{}n pushes
\item{}n times a push followed by a pop
\item{}n pops
\end{enumerate}

\paragraph{}
The second part of the proposed sequence of operations consist of n worst case pushes and n worst case pops. That makes $O(n^2)$ basic operations. We can then conclude that the two operations can't be achieved in constant amortized time.



