

\paragraph{}
We consider permutations of length $n$ .

For such permutation $a$, we define:
\begin{itemize}
\item{} the set $I(a)=\{k/ a[k]<a[k+1]\}$
\item{} $K(a)$ such that $K(a)= max(I(a))$ if $I(a)$ is not empty, $K(a)=n$ otherwise
\item{} the set $J(a)=\{l/ a[K(a)]<a[l]\}$
\item{} $L(a)$ such that $L(a) = max(J(a))$ if $J(a)$ is not empty, $L(a)=n$ otherwise
\end{itemize}

\paragraph{}
The cost of finding the next permutation is then:
\begin{itemize}
\item{} $n-K(a)-1$ to find $K(a)$
\item{} $n-K(a)-1$ at most to find the index $L(a)$
\item{} $1$ to make the swap between $a[K(a)]$ and $a[L(a)]$
\item{} $n-K(a)-1$ to reverse the list in the range of $K(a)+1,..,n-1$
\end{itemize}
The real cost to find the next permutation is then (at most) of $3*(n-K(a)-1)+1$ .

\paragraph{}
\paragraph{}
Let the potential $\Phi(a)$ be the number of indices $k$ such that $a[k]>a[k+1]$
\newline
i.e $\Phi(a)=n-\#(I(a))$ .

\paragraph{}
When we find the next permutation, we don't change any element before $K(a)$. This part doesn't contribute to any change to the potential $\Phi(a)$. The only part that can contribute in a change of potential is the part beginning at $K(a)$. By swapping $a[K(a)]$ and $a[L(a)]$, we contribute at most in a change of $+2$ in the potential. Then by reversing the end of the list, we are sure that the potential decreases by at least $n-K(a)-2$.
\paragraph{}
Thus we have $\Delta\Phi(a)\leq{4-(n-K(a))}$ .

\paragraph{}
\paragraph{}
Therefore, we have $realCost+3*\Delta\Phi=O(1)$. The operation runs in amortized constant time.